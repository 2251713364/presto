\documentclass[11pt]{article}
\usepackage{url}

\newcommand{\fdot}{$\dot{f}$}
\newcommand{\ffdot}{$f$--$\dot{f}$}
\newcommand{\PRESTO}{{\tt PRESTO}}
\newcommand{\la}{\mathrel{\hbox{\rlap{\hbox{\lower4pt\hbox{$\sim$}}}\hbox{$<$}}}}
\newcommand{\ga}{\mathrel{\hbox{\rlap{\hbox{\lower4pt\hbox{$\sim$}}}\hbox{$>$}}}}

\title{\huge \PRESTO\thanks{That's the \emph{PulsaR Exploration and
      Search TOolkit},
    according to Steve Eikenberry.}\\
  {\large Users Manual, Version 1.0}}

\author{Scott Ransom}

\begin{document}

\maketitle

\section{Introduction}

\PRESTO\ is a large suite of pulsar search and analysis software
developed almost from scratch and written primarily in ANSI C.
Written with portability, ease-of-use, and memory efficiency in mind,
it can currently handle data from the Berkeley-Caltech Pulsar Machine
(BCPM; located at the GBT), the Wideband Arecibo Pulsar
Processor (WAPP), the Parkes Multibeam system, or any individual time
series composed of single precision floating point data.  This last
format is useful for analyzing X-ray, optical, or dedispersed radio
data from any source.  Additional data formats such as phased-array
data from the GMRT and the SPIGOT card from the GBT are in the works.

The software is composed of numerous routines designed to handle three
main areas of pulsar analysis:
\begin{enumerate}
\item Data Preparation: Interference detection and removal,
  de-dispersion, barycentering.
\item Searching: Acceleration and phase-modulation (or sideband)
  searches.
\item Folding: Candidate optimization and Time-of-Arrival (TOA)
  generation.
\end{enumerate}
Many additional utilities are provided for various tasks that are
often required when working with pulsar data such as time conversions,
Fourier transforms, time series and FFT exploration, byte-swapping,
etc.  The routines are written either in C or Python\footnote{A free
  and aesthetically pleasing scripting language available from {\tt
    http://www.python.org}}.

The intent of this manual is to provide an overview of \PRESTO's
capabilities, describe in some detail the functioning of each of the
main programs, and to present various simple ``recipes'' that can be
used to accomplish typical pulsar analysis tasks.  Detailed
descriptions for most of the programs can be found by examining the
man pages located in the current directory.  Additionally, details
about command line options and parameters can be had by simply running
the program in question without any parameters.

Since the software is coded in a (relatively) modular form and is now
quite mature, adding new capabilities, such as the ability to handle a
new pulsar back-end, can be accomplished in only a few hours of
programming.  Similarly, with a large library of working routines that
cover many low-level tasks required by pulsar analysis software,
capabilities can be coded quickly and efficiently.  I welcome
suggestions for additions and/or modifications to \PRESTO\ as well as
any bug fixes you may find.  Even more welcome is real, working, code
that makes these additions, modifications, or relevant bug fixes!
Feel free to contact me at {\tt ransom@physics.mcgill.ca}.  I hope you
find \PRESTO\ useful and that \PRESTO\ finds you many new pulsars.


\section{Philosophy}

\PRESTO\ is not simple.  The reasons for this (somewhat) unfortunate
circumstance are many, but two stand out above the rest.

First, the methods that \PRESTO\ uses for many of its tasks,
especially those for the acceleration search which is it's heart, have
been optimized for memory and CPU efficiency as opposed to simplicity.
I am the first to admit that this is a controversial design choice.
It has resulted in \emph{many} hours of debugging and testing that
could have been avoided if had I chosen the path of simplicity and
re-used code from the scores of pulsar analysis packages out there.
As the lone ``true'' pulsar person at Harvard (and in the U.S.  Army)
though, I had no real access to a state-of-the-art pulsar analysis
package.  I therefore developed \PRESTO\ and many of its algorithms
from scratch as part of my PhD thesis.  Since the observational
portion of my thesis involved searching for binary millisecond pulsars
(MSPs) in globular clusters, the result is a package that allows one
to conduct coherent acceleration searches of long (i.e. many hours)
high time resolution data sets on a single ``standard'' workstation
(i.e. without many gigabytes of memory).

Second, since the software was written over the course of seven or
more years and comprises the majority of my experience with C, my
coding style has changed substantially over the time.  Many of the
low-level functions and procedures that are part of {\tt libpresto}
and are used by the majority of the programs were developed early on,
and as such, are not as I would design and write them now.  But, more
important than this, they work and are at least mostly de-bugged.
More recent code (such as {\tt accelsearch} and {\tt rfifind}) are
written (in my opinion) in a much better style but still call on many
of the older routines.

In general, though, my current philosophy is to 1) not re-invent the
wheel and 2) to use the right tool for the job.  These two rules have
resulted in me using several excellent libraries and packages in
\PRESTO.  The most important of these are {\tt FFTW} for calculating
FFTs, {\tt PGPLOT} for graphics output, {\tt TEMPO} for barycentering
calculations, and {\tt CLIG}
(\url{http://wsd.iitb.fhg.de/~kir/clighome/}) for automatically
generating a command line parser and man page for each program.  The
first rule has also resulted in me stealing code from the Parkes
Multibeam pulsar people and from Dunc Lorimer's {\tt SIGPROC}
(\url{http://www.jb.man.ac.uk/~drl/sigproc/}) in order to deal with
raw data from certain pulsar back-ends.  The second rule has resulted
in my using Python for many recently written tasks that are not CPU
intensive.  If I were re-writing \PRESTO\ from scratch, I would write
almost everything in Python and keep only the CPU-intensive cores of
the raw-data handling, de-dispersion, and search codes in C.

\section{Installation}

A quick and dirty guide to installing \PRESTO\ can be found in the top
level directory in the INSTALL file.  Slightly more in-depth
instructions are provided here.

Steps to install:

\begin{enumerate}
  
\item Install {\tt FFTW} from \url{http://www.fftw.org} in its single
  precision mode.  I highly recommend using the configuration options
  ``--enable-float --enable-type-prefix'' in order to get single
  precision libraries that have an 's' in the library names.  {\tt
    FFTW} is not actually required for \PRESTO\ since there are
  built-in FFT routines, but since {\tt FFTW} is so fast you will
  probably suffer a performance hit of 50\% or more if you don't.  If
  you happen to have Athlon based machines, I highly recommend the use
  of {\tt FFTW-GEL} as well
  (\url{http://www.complang.tuwien.ac.at/skral/fftwgel.html}).  {\tt
    FFTW-GEL} greatly speeds up short ($\la$16\,kpts) single-precision
  FFTs by using automatic SIMD vectorization of the basic {\tt FFTW}
  blocks.
  
\item Install {\tt PGPLOT} from
  \url{http://www.astro.caltech.edu/~tjp/pgplot/}.  Compile the C
  bindings as well as the standard library and insure that you compile
  in support for the postscript devices and the X-window driver.  For
  those architectures that support shared libraries, I highly
  recommend following the instruction for generating a shared version
  of libcpgplot.  It is useful (but not required) to set the
  PGPLOT\_DIR environment variable as well.
  
\item Install {\tt TEMPO} and the various ephemerides from
  \url{http://pulsar.princeton.edu/tempo/index.html}.  Make sure to
  set the TEMPO environment variable as described in the installation
  instructions.

\item Install {\tt GLIB} from
  \url{ftp://ftp.gnome.org/pub/GNOME/stable/sources/glib/}.
  
\item Set an environment variable called PRESTO that points to the top
  level directory in the \PRESTO\ distribution.
  
\item cd to \$PRESTO/src.  Check and modify the Makefile for your
  machine of choice.  Insure that the library and include file
  directories are correct for {\tt FFTW}, {\tt PGPLOT}, {\tt GLIB},
  and {\tt TEMPO}.
  
\item Get the some memory and cache information for your machine.  On
  Linux machines the easiest ways are using {\tt free} and {\tt cat
    /proc/cpuinfo}. In general, on't worry about this too much.  The
  only number that is truly important is how much memory you would
  like to allow \PRESTO\ to use.  This affects the largest in-core FFT
  you can compute before the FFT routine switches to an out-of-core
  (and hence \emph{much} slower) algorithm.  Do a {\tt make prep}
  which will ask you for the memory information and then generate an
  include file in \$PRESTO/include.
  
\item If you are using {\tt FFTW}, do a {\tt make makewisdom}.  This
  gets {\tt FFTW} acquainted with your system.  It is best if you are
  the only user on the machine when you run this, as it is very CPU
  intensive and may take a while.
  
\item Do a {\tt make} in order to make all of the executables.
  
\item The required libraries and miscellaneous files will be located
  in \$PRESTO/lib and links to the executables will be located in
  \$PRESTO/lib.  You may copy or move the executables wherever you
  like, but the library files should stay put.  (That's why you define
  the PRESTO environment variable -- so the routines can find them).
  
\item If you want to save some disk space, do a {\tt make clean} in
  the 'src' directory.  This will leave the libraries and binaries in
  their respective directories but will get rid of all the object
  files.
  
\item Go find pulsars.

\end{enumerate}

\end{document}
